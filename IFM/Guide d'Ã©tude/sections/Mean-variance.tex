\chapter[Mean-Variance Portfolio Theory]{Mean-Variance Portfolio Theory (10\% à 15\%)}
%\setlist{leftmargin=*}

\subsection{Information}

\begin{distributions}[Objective]
The Candidate will understand the assumptions of mean-variance portfolio theory and its principal results.
\end{distributions}

\begin{outcomes}[Learning outcomes]
The Candidate will be able to:
		\begin{enumerate}[label = \alph*)]
		\item	Explain the mathematics and summary statistics of portfolios.
			\begin{knowledge}[]
			\begin{itemize}
			\item	Calculate the risk and return of an asset, given appropriate inputs.
			\item	Calculate the risk and expected return of a portfolio of many risky assets, given 
				\begin{itemize}
				\item	the expected return, 
				\item	volatility, and 
				\item	correlation of returns of the individual assets.
				\end{itemize}
			\end{itemize}
			\end{knowledge}
		\item	Perform mean-variance analysis.
			\begin{knowledge}[]
			\begin{itemize}
			\item	Understand the mean-standard deviation diagram and the resulting efficient market frontier.
			\item	Calculate the optimal portfolio and determine the location of the capital market line.
			\item	Understand how portfolio risk can be reduced through diversification across multiple securities or across multiple asset classes.
			\end{itemize}
			\end{knowledge}
		\end{enumerate}
\end{outcomes}

\begin{ASM_chapter}[Related lessons ASM]
\begin{itemize}
	\item	\nameref{L.-5}
\end{itemize}
\end{ASM_chapter}

\begin{YTB_vids}[Vidéos YouTube]
\begin{itemize}
	\item	
\end{itemize}
\end{YTB_vids}

\subsection{Résumés des chapitres}

\begin{CHPT_SUMM_AUTO}[label = {L.-5}]{5. Mean-Variance Portfolio Theory}
	\begin{itemize}
		\item	
	\end{itemize}
\end{CHPT_SUMM_AUTO}

\subsection{Notes sur les vidéos YouTube}

%\begin{YTB_SUMM}[label = {}]{}
%\begin{itemize}
%	\item	
%\end{itemize}
%\end{YTB_SUMM}
