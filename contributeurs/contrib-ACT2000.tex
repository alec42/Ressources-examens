\begin{contrib}{MAS-I: Modern Actuarial Statistics I (ACT-2000, ACT-2003, ACT-2005)}
\begin{description}
	\item[aut., cre.] Alec James van Rassel
\end{description}

\textbf{\underline{Référence (manuels, YouTube, notes de cours)}}
\begin{description}
	\item[src.]	Tse, Y., Nonlife Actuarial Models, Theory Methods and Evaluation, Cambridge University Press, 2009.
	\item[src.]	Hogg, R.V.; McKean, J.W.; and Craig, A.T., Introduction to Mathematical Statistics, 7th Edition, Prentice Hall, 2013.
	\item[src.]	Weishaus, A., CAS Exam MAS-I, Study Manual, 1st Edition, Actuarial Study Materials, 2018.
	\item[src.]	Starmer, J. (2015). StatQuest. Retrieved from https://statquest.org/.
	\item[src.]	Luong, A., ACT-2000 : Analyse statistique des risques actuariels, Université Laval, Québec (QC).
	\item[src.]	Côté, M.-P., ACT-2000 : Analyse statistique des risques actuariels, Université Laval, Québec (QC).
\end{description}

\textbf{\underline{Contributeurs}}
\begin{description}
	\item[pfr.]	Sharon van Rassel
	\item[pfr.]	Louis-Philippe Vignault
	\item[pfr.]	Philippe Morin
\end{description}
\end{contrib}

\begin{distributions}[Cours reliés]
\begin{description}
	\item[ACT-2000]	Analyse statistique des risques actuariels
	\item[ACT-2003]	Modèles linéaires en actuariat
	\item[ACT-2005]	Mathématiques actuarielles IARD I
	\item[ACT-2009]	Processus stochastiques
\end{description}

En partie : mathématiques actuarielles vie I (\textbf{ACT-2004}), séries chronologiques (\textbf{ACT-2010}), introduction à l'actuariat II (\textbf{ACT-2001}) et méthodes numériques (\textbf{ACT-2002}).
\end{distributions}
