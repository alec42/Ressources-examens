\chapter[Topic: Bonds]{Topic: Bonds (10\%-20\%)}

\subsection{Information}

\begin{distributions}[Objective]
The Candidate will understand key concepts concerning bonds, and how to perform related calculations.
\end{distributions}

\begin{outcomes}[Learning outcomes]
The candidate will be able to:
\begin{enumerate}[label = \alph*)]
	\item	Define and recognize the \textit{definitions} of the following terms:
		\begin{multicols*}{2}
		\begin{itemize}[leftmargin = *]
		\item	Price;
		\item	Book value;
		\item	Amortization of premium;
		\item	Accumulation of discount;
		\item	Redemption value;
		\item	Par value / Face value;
		\item	Yield rate;
		\item	Coupon;
		\item	Coupon rate;
		\item	Term of bond;
		\item	Callable / Non-callable.
		\end{itemize}
		\end{multicols*}
	\item	Given sufficient partial information about the items listed below, calculate any of the remaining items:
		\begin{itemize}[leftmargin = *]
		\item	Price, book value, amortization of premium, accumulation of discount;
		\item	Redemption value, face value;
		\item	Yield rate;
		\item	Coupon, coupon rate;
		\item	Term of bond, point in time that a bond has a given book value, amortization of premium, or accumulation of discount.
		\end{itemize}
\end{enumerate}
\end{outcomes}

\begin{ASM_chapter}[Related lessons ASM]
Section 7: Bonds
\begin{itemize}
	\item	\nameref{L.-XX}
\end{itemize}
\end{ASM_chapter}

\begin{YTB_vids}[Vidéos YouTube]
\begin{itemize}
	\item	
\end{itemize}
\end{YTB_vids}

\subsection{Résumés des chapitres}

\begin{CHPT_SUMM_AUTO}[label = {L.-XX}]{XX. Title-of-ASM-chapter}
	\begin{itemize}
		\item	
	\end{itemize}
\end{CHPT_SUMM_AUTO}

\subsection{Notes sur les vidéos YouTube}

%\begin{YTB_SUMM}[label = {}]{}
%\begin{itemize}
%	\item	
%\end{itemize}
%\end{YTB_SUMM}