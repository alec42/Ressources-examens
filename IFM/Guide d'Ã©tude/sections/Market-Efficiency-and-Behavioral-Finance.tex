\chapter[Market Efficiency and Behavioral Finance]{Market Efficiency and Behavioral Finance (5\% à 10\%)}

\subsection{Information}

\begin{distributions}[Objective]
The Candidate will understand the notion of efficient markets and explain why market participants may make irrational systematic errors, leading to market inefficiencies.
\end{distributions}

\begin{outcomes}[Learning outcomes]
The Candidate will be able to:
\begin{enumerate}[label = \alph*)]
	\item	Explain the three forms of the efficient market hypothesis (EMH).
		\begin{knowledge}[]
		\begin{itemize}
		\item	Understand the definition of efficient markets, and distinguish between the strong, semi-strong, and weak versions of the EMH.
		\item	Identify empirical evidence for or against each form of the EMH.
		\end{itemize}
		\end{knowledge}
	\item	Explain the main findings of behavioral finance.
		\begin{knowledge}[]
		\begin{itemize}
		\item	Identify empirical examples of market anomalies that show results contrary to the EMH.
		\item	Understand how asset prices, especially in times of uncertainty and high volatility, can deviate significantly from their fundamental values.
		\end{itemize}
		\end{knowledge}
	\end{enumerate}
\end{outcomes}

\begin{ASM_chapter}[Related lessons ASM]
\begin{itemize}
	\item	\nameref{L.-4}
	\item	\nameref{L.-8}
\end{itemize}
\end{ASM_chapter}

\begin{YTB_vids}[Vidéos YouTube]
\begin{itemize}
	\item	
\end{itemize}
\end{YTB_vids}

\subsection{Résumés des chapitres}

\begin{CHPT_SUMM_AUTO}[label = {L.-4}]{4. Efficient Markets Hypothesis (EMH)}
	\begin{itemize}
		\item	
	\end{itemize}
\end{CHPT_SUMM_AUTO}

\subsection{Notes sur les vidéos YouTube}

%\begin{YTB_SUMM}[label = {}]{}
%\begin{itemize}
%	\item	
%\end{itemize}
%\end{YTB_SUMM}
