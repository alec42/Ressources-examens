\chapter[Construction and Selection of Parametric Models]{Construction and Selection of Parametric Models (20\%-30\%)}

\subsection{Information}

\begin{distributions}[Objective]
The Candidate will understand and be able to construct and estimate parameters for parametric models.
\end{distributions}

\begin{outcomes}[Learning outcomes]
The candidate will be able to:
\begin{enumerate}[label = \alph*), leftmargin = *]
	\item	Estimate the parameters (for severity, frequency, and aggregate distributions) using Maximum Likelihood Estimation for:
		\begin{itemize}[leftmargin = *]
		\item	Complete, individual data.
		\item	Complete, grouped data.
		\item	Truncated or censored data.
		\end{itemize}
	\item	Estimate the variance of the estimators and construct confidence intervals.
	\item	Use the delta method to estimate the variance of the maximum likelihood estimator of a function of the parameter(s).
	\item	Estimate the parameters (for severity, frequency, and aggregate distributions) using Bayesian Estimation.
	\item	Perform model selection using:
		\begin{itemize}[leftmargin = *]
		\item	Graphical procedures.
		\item	Hypothesis tests (incl. Chi-square goodness-of-fit, Kolmogorov-Smirnov and Likelihood ratio (LRT) tests).
		\item	Score-based approaches (incl. Schwarz Bayesian Criterion (SBC), Bayesian Information Criterion (BIC) and Akaike Information Criterion (AIC)).
		\end{itemize}
\end{enumerate}
\end{outcomes}

\begin{ASM_chapter}[Related lessons ASM]
Section X: Interest 
\begin{itemize}[leftmargin = *]
	\item	\nameref{L.-1a}
\end{itemize}
\end{ASM_chapter}

\subsection{Chapter summaries}

\begin{CHPT_SUMM_AUTO}[label = {L.-1a}]{1a. Basic}

\end{CHPT_SUMM_AUTO}
