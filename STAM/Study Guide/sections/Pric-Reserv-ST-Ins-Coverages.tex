\chapter[Pricing and Reserving for Short-Term Insurance Coverages]{Pricing and Reserving for Short-Term Insurance Coverages (15\%-25\%)}

\subsection{Information}

\begin{distributions}[Objective]
The Candidate will be able to use basic methods to calculate premiums and reserves for short-term insurance coverages.
\end{distributions}

\begin{outcomes}[Learning outcomes]
The candidate will be able to:
\begin{enumerate}[label = \alph*), leftmargin = *]
	\item	Explain the role of rating factors and exposure.
	\item	Describe the different forms of experience rating.
	\item	Describe and apply techniques for estimating unpaid losses from a run-off triangle, using the following methods:
		\begin{itemize}[leftmargin = *]
		\item	Chain ladder.
		\item	Average cost per claim.
		\item	Bornhuetter-Ferguson 
		\end{itemize}
	\item	Describe the underlying statistical models for the methods in (c).
	\item	Calculate premiums using the pure premium and loss ratio methods.
\end{enumerate}
\end{outcomes}

\begin{ASM_chapter}[Related lessons ASM]
Section X: Interest 
\begin{itemize}[leftmargin = *]
	\item	\nameref{L.-1a}
\end{itemize}
\end{ASM_chapter}

\subsection{Chapter summaries}

\begin{CHPT_SUMM_AUTO}[label = {L.-1a}]{1a. Basic}

\end{CHPT_SUMM_AUTO}
