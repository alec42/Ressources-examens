\chapter[Aggregate Models]{Aggregate Models (2.5\%-7.5\%)}

\subsection{Information}

\begin{distributions}[Objective]
The Candidate will understand and be able to perform calculations with aggregate models.
\end{distributions}

\begin{outcomes}[Learning outcomes]
The candidate will be able to (for aggregate risk models):
\begin{enumerate}[label = \alph*), leftmargin = *]
	\item	Define collective and individual risk models and calculate their expectation and variance.
	\item	Use the normal distribution to approximate the aggregate distribution.
	\item	Use the recursive formula to calculate the values of the collective risk models with discrete distributions of severities.
	\item	Calculate the expected aggregate payments in the presence of an aggregate deductible.
	\item	Evaluate the effect of the coverage modifications on the expected aggregate payments.
	\item	Perform the exact calculation of aggregate loss distribution in case of the normal distribution of severities, exponential and gamma (Erlang) distribution of severities and a compound model with negative binomial frequency and exponential distribution of severities.s
\end{enumerate}
\end{outcomes}

\begin{ASM_chapter}[Related lessons ASM]
Section X: Interest 
\begin{itemize}[leftmargin = *]
	\item	\nameref{L.-1a}
\end{itemize}
\end{ASM_chapter}

\subsection{Chapter summaries}

\begin{CHPT_SUMM_AUTO}[label = {L.-1a}]{1a. Basic}

\end{CHPT_SUMM_AUTO}
