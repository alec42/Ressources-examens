\chapter[Severity Models]{Severity Models (10\%-15\%)}

\subsection{Information}

\begin{distributions}[Objective]
The Candidate will understand and be able to perform calculations with commonly used severity models.
\end{distributions}
\begin{outcomes}[Learning outcomes]
The candidate will be able to:
\begin{enumerate}[label = \alph*), leftmargin = *]
	\item	\textcolor{ao(english)}{Calculate}:
		\begin{itemize}[leftmargin = *]
		\item	moments,
		\item	percentiles, and
		\item	generating functions.
		\end{itemize}
	\item	\textcolor{babyblue}{Describe} how changes in the parameters affect the distribution.
	\item	\textcolor{babyblue}{Recognize} classes of distributions (incl. Extreme Value distributions), and \textcolor{emerald}{their relationships}.
	\item	\textcolor{amaranth}{Create} new distributions by:
		\begin{itemize}[leftmargin = *]
		\item	multiplication by a constant, 
		\item	raising to a power, 
		\item	exponentiation, and 
		\item	mixing.
		\end{itemize}
	\item	\textcolor{babyblue}{Identify} the applications to which each distribution may apply and \textcolor{amber}{explain} why.
	\item	\textcolor{ao(english)}{Apply} the distribution to an application, given the parameters.
	\item	\textcolor{emerald}{Compare} two distributions based on various characteristics of their tails, including :
		\begin{itemize}[leftmargin = *]
		\item	moments,
		\item	ratios of moments, 
		\item	limiting tail behaviour, 
		\item	hazard rate function, and 
		\item	mean excess function.
		\end{itemize}
\end{enumerate}
\end{outcomes}

\begin{ASM_chapter}[Related lessons ASM]
Section X: Interest 
\begin{itemize}[leftmargin = *]
	\item	\nameref{L.-1a}
\end{itemize}
\end{ASM_chapter}

\subsection{Chapter summaries}

\begin{CHPT_SUMM_AUTO}[label = {L.-1a}]{1a. Basic}

\end{CHPT_SUMM_AUTO}
