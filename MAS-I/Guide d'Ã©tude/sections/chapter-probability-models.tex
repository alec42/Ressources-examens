\chapter[Probability models (Stochastic Processes and Survival Models)]{Probability models (Stochastic Processes and Survival Models) (20\% à 35\%)}

\subsection{Information}

\begin{distributions}[Description]
Notes du descriptif principal:
\begin{itemize}
	\item	Stochastic processes
	\item	Survival models
		\begin{itemize}
		\item	Covered in depth as part of probability modeling in generic terms;
		\end{itemize}
	\item	Markov Chains
		\begin{itemize}
		\item	Provide the means to model how an entity can move through different states;
		\end{itemize}
	\item	Simplified version of life contingencies
		\begin{itemize}
		\item	Life contingencies problems can be viewed as discounted cash flow problems which include thee effect of probability of payment;
		\item	Covered through a study note which link the generic survival model concepts to a subset of life actuarial concepts;
		\item	This study note illustrates how to calculate annuities or single premium insurance amounts;
		\end{itemize}
\end{itemize}
\tcbline
Notes de la sous-section:
\begin{itemize}
	\item	Résoudre des problèmes de processus aléatoires;
	\item	Identifier les probabilités et distributions associées avec ces processus;
		\begin{itemize}
		\item	Particulièrement, être capable d'utiliser un processus de Poisson dans ces applications;
		\end{itemize}
	\item	Les modèles de survie sont une rallonge aux modèles de probabilité de processus stochastiques;
		\begin{itemize}
		\item	En lieu, on estime la vie futur d'une entité avec quelques suppositions sur la distribution de la vraisemblance de survie;
		\end{itemize}
	\item	Chaines de Markov utiles pour modéliser la mobilité entre états dans un processus et souligner les modèles Bayésien MCMC sous-jacent;
	\item	La simulation est incluse puisqu'elle peut s'avérer essentielle pour arriver à une solution de problème complexe;
\end{itemize}
\end{distributions}


\begin{outcomes}[Learning objectives]
\begin{enumerate}
	\item	Understand and apply the properties of Poisson processes;
		\begin{itemize}
		\item	For increments in the homogeneous case;
		\item	For interval times in the homogeneous case;
		\item	For increments in the non-homogeneous case;
		\item	Resulting from special types of events in the Poisson process;
		\item	Resulting from sums of independent Poisson processes;
		\end{itemize}
	\begin{knowledge}[Knowledge Statements]
	\begin{enumerate}[label = \alph*.]
		\item	Poisson process;
		\item	Non-homogeneous Poisson process;
		\item	Memoryless property of Exponential and Poisson;
		\item	Relationship between Exponential and Gamma;
		\item	Relationship between Exponential and Poisson;
	\end{enumerate}
	\end{knowledge}
\tcbline
	\item	For any Poisson process and the inter-arrival and waiting distributions associated with the Poisson process, calculate:
		\begin{itemize}
		\item	Expected values;
		\item	Variances;
		\item	Probabilities;
		\end{itemize}
	\begin{knowledge}[Knowledge Statements]
	\begin{enumerate}[label = \alph*.]
		\item	Probability calculations for Poisson process
		\item	Conditional distribution of arrival times;
		\item	Splitting grouped Poisson rate to subsets of population using probability distribution;
		\item	Conditional distribution of events by category within a group within a certain time period;
	\end{enumerate}
	\end{knowledge}
\tcbline
	\item	For a compound Poisson process, calculate moments associated with the value of the process at a given time;
	\begin{knowledge}[Knowledge Statements]
	\begin{enumerate}[label = \alph*.]
		\item	Compound Poisson process mean and variance;
		\item	Normal approximation and hypothesis testing;
	\end{enumerate}
	\end{knowledge}
\tcbline
	\item	Apply the Poisson process concepts to calculate the hazard function and related survival model concepts;
		\begin{itemize}
		\item	Relationship between hazard rate, probability density function and cumulative distribution function;
		\item	Effect of memoryless nature of Poisson distribution on survival time estimation;
		\end{itemize}
	\begin{knowledge}[Knowledge Statements]
	\begin{enumerate}[label = \alph*.]
		\item	Failure time random variables;
		\item	Cumulative distribution functions;
		\item	Survival functions;
		\item	Probability density functions;
		\item	Hazard functions and relationship to Exponential distribution;
		\item	Relationships between failure time random variables in the functions above;
		\item	Greedy algorithms;
	\end{enumerate}
	\end{knowledge}
\tcbline
	\item	Given the joint distribution of more than one source of failure in a system (or life) and using Poisson Process assumptions:
		\begin{itemize}
		\item	Calculate probabilities and moments associated with functions of these random variables' variances;
		\item	Understand differences between a series system (joint life) and parallel system (last survivor) when calculating expected time to failure or probability of failure by a certain time;
		\item	Understand the effect of multiple sources of failure (multiple decrement) on expected system time to failure (expected lifetime);
		\end{itemize}
	\begin{knowledge}[Knowledge Statements]
	\begin{enumerate}[label = \alph*.]
		\item	Joint distribution of failure times;
		\item	Probabilities and moments;
		\item	Time until failure of the system (life);
		\item	Time until failure of the system (life) from a specific cause;
		\item	Time until failure of the system (life) for parallel or series systems with multiple components;
		\item	Paths that lead to parallel or series system failures for systems with multiple components;
		\item	Relationship between failure time and minimal path and minimal cut sets;
		\item	Bridge system and defining path to failure;
		\item	Random graphs and defining path to failure;
		\item	Effect of multiple sources of failure (multiple decrements) on failure time calculations (competing risk);
		\item	Non-uniform probability of component failure (multiple decrement);
		\item	Method of inclusion and exclusion as applied to failure time estimates;
		\item	Expected system lifetime as function of component lifetime and properties of expected lifetime estimates;
		\item	Linkage between reliability function for a system and future expected lifetime;
	\end{enumerate}
	\end{knowledge}
\tcbline
	\item	For discrete Markov Chains under both homogeneous and non-homogenous states:	
		\begin{itemize}
		\item	Definition of a Markov Chain;
		\item	Chapman-Kolmogorov Equations for $n$-step transition calculations;
		\item	Accessible states;
		\item	Ergodic Markov Chains and limiting probabilities;
		\end{itemize}
	\begin{knowledge}[Knowledge Statements]
	\begin{enumerate}[label = \alph*.]
		\item	Random Walk;
		\item	Classification of states and classes of states (absorbing, accessible, transition, irreducible, and recurrent);
		\item	Transition step probabilities;
		\item	Stationary probabilities;
		\item	Recurrent vs. transient states;
		\item	Gamblers ruin problem;
		\item	Branching processes;
		\item	Homogeneous transition probabilities;
		\item	Memoryless property of Markov Chains;
		\item	Limiting probabilities;
	\end{enumerate}
	\end{knowledge}
\tcbline
	\item	Solve Life Contingency problems using a life table in a spreadsheet as the combined result of discount, probability of payment and amount of payment vectors. Understand the linkage between the life table and the corresponding probability models;
		\begin{itemize}
		\item	Calculate annuities for discrete time;
		\item	Calculate life insurance single net premiums (or P \& C pure premiums) for discrete time;
		\item	Solve for net level premiums (\textbf{not} including fractional lives);
		\end{itemize}
	\begin{knowledge}[Knowledge Statements]
	\begin{enumerate}[label = \alph*.]
		\item	Discounted cash flow;
		\item	Relationship between annuity values and insurance premiums;
		\item	Life table linkage to probability models;
		\item	Equivalence property;
	\end{enumerate}
	\end{knowledge}
\tcbline
	\item	The candidate should be familiar with basic computer simulation methods.
		\begin{itemize}
		\item	Understand the basic framework of Monte Carlo Simulation;
		\item	Understand the mechanics of generating uniform random numbers;
		\item	Generate random numbers from a variety of distributions using the inversion method;
		\item	Be able to explain when and how to use the Acceptance-Rejection method;
		\end{itemize}
	\begin{knowledge}[Knowledge Statements]
	\begin{enumerate}[label = \alph*.]
		\item	Random Number Generation;
		\item	Uniform Random Numbers;
		\item	Inversion Method;
		\item	Acceptance-Rejection Method;
	\end{enumerate}
	\end{knowledge}
\end{enumerate}
\end{outcomes}

\begin{ASM_chapter}[Related lessons ASM]
\begin{enumerate}
	\item	\nameref{L.-1}
	\item	\nameref{L.-2}
	\item	\nameref{L.-3}
\tcbline
	\item	\nameref{L.-4}
	\item	\nameref{L.-5}
	\item	\nameref{L.-6}
	\item	\nameref{L.-7}
	\item	\nameref{L.-8}
	\item	\nameref{L.-9}
	\item	\nameref{L.-10}
	\item	\nameref{L.-11}
	\item	\nameref{L.-12}
	\item	\nameref{L.-13}
	\item	\nameref{L.-14}
	\item	\nameref{L.-15}
	\item	\nameref{L.-16}
	\item	\nameref{L.-17}
	\item	\nameref{L.-18}
	\item	\nameref{L.-19}
\tcbline
	\item	\nameref{L.-20}
	\item	\nameref{L.-21}
\tcbline
	\item	\nameref{L.-22}
	\item	\nameref{L.-23}
	\item	\nameref{L.-24}
\end{enumerate}
\end{ASM_chapter}

\begin{YTB_vids}[Vidéos YouTube]
\begin{itemize}
	\item	
\end{itemize}
\end{YTB_vids}

\subsection{Résumés des chapitres}

\subsubsection*{Probability Review}

\begin{CHPT_SUMM_AUTO_NUMB}[label = {L.-1}]{Probability Review}
Introduction to Mathematical Statistics 1 - 3, 5
	\begin{itemize}
		\item	
	\end{itemize}
\end{CHPT_SUMM_AUTO_NUMB}

\begin{CHPT_SUMM_AUTO_NUMB}[label = {L.-2}]{Parametric Distributions}
Introduction to Mathematical Statistics 2.2, 2.7
Nonlife Actuarial Models---Theory Methods and Evaluation 2.2
	\begin{itemize}
		\item	
	\end{itemize}
\end{CHPT_SUMM_AUTO_NUMB}

\begin{CHPT_SUMM_AUTO_NUMB}[label = {L.-3}]{Mixtures}
Introduction to Mathematical Statistics 3.7
Nonlife Actuarial Models---Theory Methods and Evaluation 2.3.2
	\begin{itemize}
		\item	
	\end{itemize}
\end{CHPT_SUMM_AUTO_NUMB}

\subsubsection*{Stochastic Processes}

\begin{CHPT_SUMM_AUTO_NUMB}[label = {L.-4}]{Markov Chains: Chapman-Kolmogorov Equations}
Ross 4.1 - 4.2, 4.5.1 - 4.5.2
	\begin{itemize}
		\item	
	\end{itemize}
\end{CHPT_SUMM_AUTO_NUMB}

\begin{CHPT_SUMM_AUTO_NUMB}[label = {L.-5}]{Markov Chains: Classification of States}
Ross 4.3
	\begin{itemize}
		\item	
	\end{itemize}
\end{CHPT_SUMM_AUTO_NUMB}

\begin{CHPT_SUMM_AUTO_NUMB}[label = {L.-6}]{Discrete Markov Chains: Long-Run Proportions and Limiting Probabilities}
Ross 4.4
	\begin{itemize}
		\item	
	\end{itemize}
\end{CHPT_SUMM_AUTO_NUMB}

\begin{CHPT_SUMM_AUTO_NUMB}[label = {L.-7}]{Markov Chains: Time in Transient States}
Ross 4.6
	\begin{itemize}
		\item	
	\end{itemize}
\end{CHPT_SUMM_AUTO_NUMB}

\begin{CHPT_SUMM_AUTO_NUMB}[label = {L.-8}]{Markov Chains: Branching Processes}
Ross 4.7
	\begin{itemize}
		\item	
	\end{itemize}
\end{CHPT_SUMM_AUTO_NUMB}

\begin{CHPT_SUMM_AUTO_NUMB}[label = {L.-9}]{Markov Chains: Time Reversible}
Ross 4.8
	\begin{itemize}
		\item	
	\end{itemize}
\end{CHPT_SUMM_AUTO_NUMB}

\begin{CHPT_SUMM_AUTO_NUMB}[label = {L.-10}]{Exponential Distribution}
Ross 5.2
	\begin{itemize}
		\item	
	\end{itemize}
\end{CHPT_SUMM_AUTO_NUMB}

\begin{CHPT_SUMM_AUTO_NUMB}[label = {L.-11}]{The Poisson Process: Probabilities of Events}
Ross 5.3.1 - 5.3.2
Daniel Poisson Study Note 1.1, 1.4.1
	\begin{itemize}
		\item	
	\end{itemize}
\end{CHPT_SUMM_AUTO_NUMB}

\begin{CHPT_SUMM_AUTO_NUMB}[label = {L.-12}]{The Poisson Process: Time To Next Event}
Ross 5.2, 5.3.3
Daniel Poisson Study Note 1.1.1
	\begin{itemize}
		\item	
	\end{itemize}
\end{CHPT_SUMM_AUTO_NUMB}

\begin{CHPT_SUMM_AUTO_NUMB}[label = {L.-13}]{{The Poisson Process: Thinning, or Couting Special Types of Events}}
Ross 5.3.4
Daniel Poisson Study Note 1.3.1, 1.4.3
	\begin{itemize}
		\item	
	\end{itemize}
\end{CHPT_SUMM_AUTO_NUMB}

\begin{CHPT_SUMM_AUTO_NUMB}[label = {L.-14}]{The Poisson Process: Other Characteristics}
Ross 5.2.3, 5.3.4, 5.3.5
	\begin{itemize}
		\item	
	\end{itemize}
\end{CHPT_SUMM_AUTO_NUMB}

\begin{CHPT_SUMM_AUTO_NUMB}[label = {L.-15}]{The Poisson Process: Sums and Mixtures}
Ross 5.4.3
Daniel Poisson Study Note 1.3.2, 1.3.3
	\begin{itemize}
		\item	
	\end{itemize}
\end{CHPT_SUMM_AUTO_NUMB}

\begin{CHPT_SUMM_AUTO_NUMB}[label = {L.-16}]{Compound Poisson Processes}
Ross 5.4.2
Daniel Poisson Study Note 1.2, 1.4.2
	\begin{itemize}
		\item	
	\end{itemize}
\end{CHPT_SUMM_AUTO_NUMB}

\begin{CHPT_SUMM_AUTO_NUMB}[label = {L.-17}]{Reliability: Structure Functions}
Ross 9.1 - 9.2
	\begin{itemize}
		\item	
	\end{itemize}
\end{CHPT_SUMM_AUTO_NUMB}

\begin{CHPT_SUMM_AUTO_NUMB}[label = {L.-18}]{Reliability: Probabilities}
Ross 9.3 - 9.4
	\begin{itemize}
		\item	
	\end{itemize}
\end{CHPT_SUMM_AUTO_NUMB}

\begin{CHPT_SUMM_AUTO_NUMB}[label = {L.-19}]{Reliability: Time to Failure}
Ross 9.5 - 9.6
	\begin{itemize}
		\item	
	\end{itemize}
\end{CHPT_SUMM_AUTO_NUMB}

\subsubsection*{Life Contingencies}

\begin{CHPT_SUMM_AUTO_NUMB}[label = {L.-20}]{Survival Models}
Struppeck 1, 2, 6, 7
	\begin{itemize}
		\item	
	\end{itemize}
\end{CHPT_SUMM_AUTO_NUMB}

\begin{CHPT_SUMM_AUTO_NUMB}[label = {L.-21}]{Contingent Payments}
Struppeck 3, 4, 5, 6
	\begin{itemize}
		\item	
	\end{itemize}
\end{CHPT_SUMM_AUTO_NUMB}

\subsubsection*{Simulation}

\begin{CHPT_SUMM_AUTO_NUMB}[label = {L.-22}]{Simulation---Inverse Transformation Method}
Ross 11.1, 11.2.1
	\begin{itemize}
		\item	
	\end{itemize}
\end{CHPT_SUMM_AUTO_NUMB}

\begin{CHPT_SUMM_AUTO_NUMB}[label = {L.-23}]{Simulation---Applications}
Ross 11.1, 11.2.1
	\begin{itemize}
		\item	
	\end{itemize}
\end{CHPT_SUMM_AUTO_NUMB}

\begin{CHPT_SUMM_AUTO_NUMB}[label = {L.-24}]{Simulation---Rejection Method}
Ross 11.2.2
	\begin{itemize}
		\item	
	\end{itemize}
\end{CHPT_SUMM_AUTO_NUMB}

\subsection{Notes sur les vidéos YouTube}

\begin{YTB_SUMM}[label = {SQ-BASICS-ML-INTRO}]{\href{https://www.youtube.com/watch?v=Gv9_4yMHFhI&list=PLblh5JKOoLUICTaGLRoHQDuF_7q2GfuJF&index=2&t=0s}{StatQuest: A Gentle Introduction to Machine Learning}}
\begin{itemize}
	\item	
\end{itemize}
\end{YTB_SUMM}
