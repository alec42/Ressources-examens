\chapter[Statistics]{Statistics (15\% à 30\%)}

\subsection{Information}

\begin{distributions}[Description]
Notes du descriptif principal:
\begin{itemize}
	\item	Topics which would commonly be covered in a 2-semester Probability \& Statistics sequence;
\end{itemize}
\end{distributions}

\begin{outcomes}[Learning objectives]
	\begin{enumerate}
	\item	Perform point estimation of statistical parameters using Maximum likelihood estimation (MLE). \\
			Apply criteria to estimates such as :
	\begin{multicols*}{2}
		\begin{itemize}
		\item	Consistency;
		\item	Unbiasedness;
		\item	Sufficiency;
		\item	Efficiency;
		\item	Minimum variance;
		\item	MSE;
		\end{itemize}
	\end{multicols*}
	\end{enumerate}
\end{outcomes}
\begin{knowledge}[Knowledge Statements]
	\begin{enumerate}[label = \alph*.]
	\item	Equations for MLE of mean, variance from a sample;
	\item	Estimation of mean and variance based on samples;
	\item	General equations for MLE of parameters;
	\item	Recognition of consistency property of estimators and alternative measures of consistency;
	\item	Application of criteria for measurement when estimating parameters through minimisation of variance, MSE;
	\item	Definition of statistical bias and recognition of estimators that are unbiased or biased;
	\item	Application of Rao-Cramer Lower Bound and Efficiency;
	\item	Relationship between Sufficiency and Minimum Variance;
	\item	Develop and estimate a sufficient statistic for a distribution;
	\item	Factorization Criterion for sufficiency;
	\item	Application of Rao-Cramer Lower Bound and Fisher Information;
	\item	Application of MVUE for the exponential class of distributions;
	\item	Linkage between Score Function, Fisher Information and maximum likelihood;
	\item	Method of Moments;
	\item	Percentile Matching;
	\item	Kernel Density Estimation;
	\item	Maximum Likelihood with Censoring and Truncation;
	\end{enumerate}
\end{knowledge}

\begin{outcomes}[Learning objectives]
	\begin{enumerate}
  \setcounter{enumi}{1}
	\item	Calculate parameter estimates using methods other than maximum likelihood.
	\item	Test statistical hypotheses including Type I and Type II errors using:
		\begin{itemize}
		\item	Neyman-Pearson theorem;
		\item[]	Apply Neyman-Pearson theorem to construct likelihood ratio equation;
		\item	Likelihood ratio tests;
		\item	First principles;
		\item[]	Use critical values from a sampling distribution to test means and variances;
		\end{itemize}	
	\end{enumerate}
\end{outcomes}	
\begin{knowledge}[Knowledge Statements]
	\begin{enumerate}[label = \alph*.]
	\item	Presentation of fundamental inequalities based on general assumptions and normal assumptions;
	\item	Definition of Type I and Type II errors;
	\item	Significance levels;
	\item	One-sided versus two-sided tests;
	\item	Estimation of sample sizes under normality to control for Type I and Type II errors;
	\item	Determination of critical regions;
	\item	Definition and measurement of likelihood ratio tests;
	\item	Determining parameters and testing using tabular values (from a table);
	\item	Recognizing when to apply likelihood ratio tests versus chi-square or other goodness of fit tests;
	\item	Apply paired $t$-test to two samples;
	\item	Test for difference in variance under Normal distribution between two samples through the application of $F$-test;
	\item	Test of significance of means from two samples under Normal distribution assumptions in both large and small sample cases;
	\item	Test for significance of difference in proportions between two samples under the Binomial distribution assumption in both large and small sample cases;
	\item	Application of contingency tables to test independence between effects;
	\item	Asymptotic relationship between likelihood ratio tests and the Chi-Square distribution;
	\item	Application of Neyman-Pearson theorem to Uniformly Most Powerful hypothesis tests;
	\item	Equivalence between critical regions and confidence intervals;
	\item	Kolmogorov-Smirnov test;
	\end{enumerate}
\end{knowledge}
	
\begin{outcomes}[Learning objectives]
	\begin{enumerate}
  \setcounter{enumi}{3}
	\item	For the Exponential, Gamma, Weibull, Pareto, Lognormal, Beta, and mixtures thereof:
		\begin{itemize}
		\item	Identify the applciations to Insurance claim modeling in which each distribution is used and reasons why;
		\item	Transformation of distributions; 
		\end{itemize}
	\end{enumerate}
\end{outcomes}
\begin{knowledge}[Knowledge Statements]
	\begin{enumerate}[label = \alph*.]
	\item	Frequency, severity and aggregate loss;
	\item	Common continuous distributions for modeling claim severity;
	\item	Mixing distributions;
	\item	Tail properties of claim severity;
	\item	Effects of coverage modifications including, for example: limits, deductibles, loss elimination ratios and effects of inflation;
	\end{enumerate}
\end{knowledge}

\begin{outcomes}[Learning objectives]
	\begin{enumerate}
  \setcounter{enumi}{4}
	\item	Calculate Order Statistics of a sample for a given distribution.
	\end{enumerate}
\end{outcomes}
\begin{knowledge}[Knowledge Statements]
	\begin{enumerate}[label = \alph*.]
	\item	General form for distribution of $n^{\text{th}}$ largest element of a set;
	\item	Application to a given distributional form;
	\end{enumerate}
\end{knowledge}

\begin{ASM_chapter}[Related lessons ASM]
\begin{enumerate}
  \setcounter{enumi}{24}
	\item	\nameref{L.-25}
	\item	\nameref{L.-26}
	\item	\nameref{L.-27}
	\item	\nameref{L.-28}
	\item	\nameref{L.-29}
	\item	\nameref{L.-30}
	\item	\nameref{L.-31}
	\item	\nameref{L.-32}
	\item	\nameref{L.-33}
	\item	\nameref{L.-34}
	\item	\nameref{L.-35}
	\item	\nameref{L.-36}
	\item	\nameref{L.-37}
	\item	\nameref{L.-38}
	\item	\nameref{L.-39}
	\item	\nameref{L.-40}
	\item	\nameref{L.-41}
\end{enumerate}
\end{ASM_chapter}

\begin{YTB_vids}[Vidéos YouTube]
\begin{itemize}
	\item	
\end{itemize}
\end{YTB_vids}

\begin{distributions}[Likely Questions]
\begin{itemize}
	\item	Question where we calculate the sample variance with the STAT function of the calculator;
		\begin{itemize}
		\item	MAS-I F19, \# 15	;
		\end{itemize}
\end{itemize}
\end{distributions}

\subsection{Résumés des chapitres}

\begin{CHPT_SUMM_AUTO_NUMB}[label = {L.-25}]{Estimator Quality}
Introduction to Mathematical  4.1.3, 5.1, 7.1\\

\tcbline

Overview of various basic functions to evaluate the quality of an estimator including:
\begin{enumerate}[leftmargin = *]
	\item	The Bias of an estimator (incl. asymptotically unbiased).
		\begin{itemize}[leftmargin = *]
		\item	Shows the sample mean $\bar{x}$ is an unbiased estimator of the true mean $\mu$.
		\item	Shows the sample variance $s^{2} = \frac{\sum (x_{i} - \bar{x})^{2}}{n - 1}$ is an unbiased estimator of the true variance $\sigma^{2}$.
		\item	Shows the empirical variance $\hat{\sigma}^{2} = \frac{\sum (x_{i} - \bar{x})^{2}}{n}$ is an biased estimator of the true variance $\sigma^{2}$.	\\
				However, with a bias of $\frac{n - 1}{n}\sigma^{2} - \sigma^{2} = -\frac{\sigma^{2}}{n}$, it is \textit{asymptotically} unbiased.
		\item	Gives the boiling water feet-freezer head analogy for why the bias is not sufficient to assess the quality of an estimator.
		\end{itemize}
	\item	The consistency of an estimator.
		\begin{itemize}[leftmargin = *]
		\item	As the number of observations $n$ on which an estimator $\hat{\theta}_{n}$ is based increases, if both the bias and variance of $\hat{\theta}_{n}$ go to 0, we can say it is a consistent estimator.	\\
				Although this condition is sufficient, it is not necessary.
		\item	For example, the sample mean $\bar{x}$ is a consistent estimator of the true mean for both the Gamma and Normal distributions.	\\
				However, a Pareto distribution with $\alpha \leq 2$ has an infinite variance; therefore, the variance of the estimator is infinite and $\bar{x}$ is not a consistent estimator.
		\end{itemize}
	\item	The efficiency of an estimator.
		\begin{itemize}[leftmargin = *]
		\item	Shows the relative efficiency of one estimator to another.
		\end{itemize}
	\item	The Mean Square Error (MSE) of an estimator.
		\begin{itemize}[leftmargin = *]
		\item	Shows both how it's defined as the variance of the estimator's predictions of the parameter and the relation with bias and variance.
		\end{itemize}
\end{enumerate}

We then combine these concepts to define the \textbf{uniformly minimum variance unbiased estimator (UMVUE)} which has a smaller variance, for any true value $\theta$, than any other \underline{unbiased} estimator.
\end{CHPT_SUMM_AUTO_NUMB}

\begin{CHPT_SUMM_AUTO_NUMB}[label = {L.-26}]{Kernel Density Estimation}
Nonlife Actuarial Models: Theory Methods and Evaluation 11.1
\tcbline
\begin{itemize}
		\item	
	\end{itemize}
\end{CHPT_SUMM_AUTO_NUMB}

\begin{CHPT_SUMM_AUTO_NUMB}[label = {L.-27}]{Method of Moments}
Nonlife Actuarial Models: Theory Methods and Evaluation 12.1.1
\tcbline
Three types of incomplete data:
\begin{description}
	\item[Grouped data]	Given a set of intervals and told how many observations are in each.
	\item[Censored data]	Given that the value of an observation is in a range, but not given the exact value. For example, a policy with a limit of 10'000\$.
	\item[Truncated data]	Given the value of an observation only when it is in a certain range; typically, only above or below a certain number. For example, a policy with a deductible of 100\$ has no recorded losses of 100\$ or less.
\end{description}
\tcbline
To fit a $k$ parameter distribution, we set equal the $k$ first sample moments $\hat{\mu}_{k}$ to the $k$ first raw moments $\mu_{k}'$ of the distribution. \\	
We may match the variances instead of the second moments but we match the biased empirical variance  $\hat{\sigma}^{2}$ by default and not $s^{2}$.

\begin{align*}
	 \hat{\mu}_{j}	&=	 \mu_{j}', \; j = 1, \dots, k
\end{align*}
\tcbline
If data is censored at $u$, $\hat{\mu}_{k}	=	 \text{E}[\min(X; u)^{k}]$.\\
If data is truncated at $d$, $\hat{\mu}_{k}	=	 \text{E}[X^{k} | X > d]$.
\end{CHPT_SUMM_AUTO_NUMB}

\begin{CHPT_SUMM_AUTO_NUMB}[label = {L.-28}]{Percentile Matching}
Nonlife Actuarial Models: Theory Methods and Evaluation 11.1.1, 12.1.2

\tcbline


	\begin{itemize}
		\item	
	\end{itemize}
\end{CHPT_SUMM_AUTO_NUMB}

\begin{CHPT_SUMM_AUTO_NUMB}[label = {L.-29}]{Maximum Likelihood Estimators}
Introduction to Mathematical Statistics 4.1, 6.1
Nonlife Actuarial Models: Theory Methods and Evaluation 10.2, 12.3
	\begin{itemize}
		\item	
	\end{itemize}
\end{CHPT_SUMM_AUTO_NUMB}

\begin{CHPT_SUMM_AUTO_NUMB}[label = {L.-30}]{Maximum Likelihood Estimators---Special Techniques}
Introduction to Mathematical Statistics 4.1, 6.1
Nonlife Actuarial Models: Theory Methods and Evaluation 10.2, 12.3
	\begin{itemize}
		\item	
	\end{itemize}
\end{CHPT_SUMM_AUTO_NUMB}

\begin{CHPT_SUMM_AUTO_NUMB}[label = {L.-31}]{Variance of Maximum Likelihood Estimator}
Introduction to Mathematical Statistics 6.2, 6.5
	\begin{itemize}
		\item	
	\end{itemize}
\end{CHPT_SUMM_AUTO_NUMB}

\begin{CHPT_SUMM_AUTO_NUMB}[label = {L.-32}]{Sufficient Statistics}
Introduction to Mathematical Statistics 7
	\begin{itemize}
		\item	
	\end{itemize}
\end{CHPT_SUMM_AUTO_NUMB}

\begin{CHPT_SUMM_AUTO_NUMB}[label = {L.-33}]{Hypothesis Testing}
Introduction to Mathematical Statistics 4.5, 4.6
	\begin{itemize}
		\item	
	\end{itemize}
\end{CHPT_SUMM_AUTO_NUMB}

\begin{CHPT_SUMM_AUTO_NUMB}[label = {L.-34}]{Confidence Intervals and Sample Size}
Introduction to Mathematical Statistics 4.5, 4.6
	\begin{itemize}
		\item	
	\end{itemize}
\end{CHPT_SUMM_AUTO_NUMB}

\begin{CHPT_SUMM_AUTO_NUMB}[label = {L.-35}]{Confidence Intervals for Means}
Introduction to Mathematical Statistics 4.2
	\begin{itemize}
		\item	
	\end{itemize}
\end{CHPT_SUMM_AUTO_NUMB}

\begin{CHPT_SUMM_AUTO_NUMB}[label = {L.-36}]{Kolmogorov-Smirnov Tests}
Nonlife Actuarial Models: Theory Methods and Evaluation 13.2.1
	\begin{itemize}
		\item	
	\end{itemize}
\end{CHPT_SUMM_AUTO_NUMB}

\begin{CHPT_SUMM_AUTO_NUMB}[label = {L.-37}]{Chi Square Tests}
Introduction to Mathematical 4.7
Nonlife Actuarial Models: Theory Methods and Evaluation 13.2.3
	\begin{itemize}
		\item	
	\end{itemize}
\end{CHPT_SUMM_AUTO_NUMB}

\begin{CHPT_SUMM_AUTO_NUMB}[label = {L.-38}]{Confidence Intervals for Variances}
Introduction to Mathematical Statistics 8.3
	\begin{itemize}
		\item	
	\end{itemize}
\end{CHPT_SUMM_AUTO_NUMB}

\begin{CHPT_SUMM_AUTO_NUMB}[label = {L.-39}]{Uniformly Most Powerful Critical Regions}
Introduction to Mathematical Statistics 8.1 - 8.2
	\begin{itemize}
		\item	
	\end{itemize}
\end{CHPT_SUMM_AUTO_NUMB}

\begin{CHPT_SUMM_AUTO_NUMB}[label = {L.-40}]{Likelihood Ratio Tests}
Introduction to Mathematical Statistics 8.3
	\begin{itemize}
		\item	
	\end{itemize}
\end{CHPT_SUMM_AUTO_NUMB}

\begin{CHPT_SUMM_AUTO_NUMB}[label = {L.-41}]{Q.-Q. Plots}
Introduction to Mathematical Statistics 4.4
Larsen Study Note
	\begin{itemize}
		\item	
	\end{itemize}
\end{CHPT_SUMM_AUTO_NUMB}

\subsection{Notes sur les vidéos YouTube}

\begin{YTB_SUMM}[label = {SQ-BASICS-ML-INTRO}]{\href{https://www.youtube.com/watch?v=Gv9_4yMHFhI&list=PLblh5JKOoLUICTaGLRoHQDuF_7q2GfuJF&index=2&t=0s}{StatQuest: A Gentle Introduction to Machine Learning}}
\begin{itemize}
	\item	
\end{itemize}
\end{YTB_SUMM}
