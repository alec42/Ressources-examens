\chapter[Topic: Determinants of Interest Rates]{Topic: Determinants of Interest Rates (0-10\%)}

\subsection{Information}

\begin{distributions}[Objective]
The Candidate will understand key concepts concerning the determinants of interest rates, the components of interest, and how to perform related calculations.
\end{distributions}

\begin{outcomes}[Learning outcomes]
The candidate will be able to:
\begin{enumerate}[label = \alph*)]
	\item	Define and recognize the \textit{definitions} of the following terms:
		\begin{multicols*}{2}
		\begin{itemize}[leftmargin = *]
		\item	Real risk-free rate;
		\item	Inflation rate;
		\item	Default risk premium;
		\item	Liquidity premium;
		\item	Maturity risk premium.
		\end{itemize}
		\end{multicols*}
	\item	Explain how the components of interest rates apply in various contexts, such as:
		\begin{itemize}[leftmargin = *]
		\item	Commercial loans;
		\item	Mortgages;
		\item	Credit cards;
		\item	Bonds;
		\item	Government securities.
		\end{itemize}
	\item	Explain the \textbf{roles} of the Federal Reserve and the FOMC in carrying out \textit{fiscal} policy and \textit{monetary} policy and the \textbf{tools} used thereby including:
		\begin{itemize}
		\item	Targeting the federal funds rate;
		\item	Setting reserve requirements;
		\item	Setting the discount rate.
		\end{itemize}
	\item	Explain the theories of why interest rates differ by term, including :
		\begin{itemize}
		\item	Liquidity preference (opportunity cost);
		\item	Expectations;
		\item	Preferred habitat; 
		\item	Market segmentation.
		\end{itemize}
	\item	Explain how interest rates differ from one country to another (e.g., U.S. vs. Canada);
	\item	In the context of loans with and without inflation protection:
		\begin{itemize}
		\item	\textbf{Identify} the \textit{real} interest and the \textit{nominal} interest rate;
		\item	\textbf{Calculate} the effect of changes in inflation on loans with inflation protection.
		\end{itemize}
\end{enumerate}
\end{outcomes}

\begin{ASM_chapter}[Related lessons ASM]
Section 9: Determinants of Interest Rates
\begin{itemize}
	\item	\nameref{L.-9a}
	\item	\nameref{L.-9b}
	\item	\nameref{L.-9c}
	\item	\nameref{L.-9d}
	\item	\nameref{L.-9e}
	\item	\nameref{L.-9f}
	\item	\nameref{L.-9g}
	\item	\nameref{L.-9h}
	\item	\nameref{L.-9i}
\end{itemize}
\end{ASM_chapter}

\subsection{Chapter summaries}

\begin{CHPT_SUMM_AUTO}[label = {L.-9a}]{9a. What is Interest?}
	\begin{itemize}
		\item	
	\end{itemize}
\end{CHPT_SUMM_AUTO}

\begin{CHPT_SUMM_AUTO}[label = {L.-9b}]{9b. Quotation Bases for Interest Rates}
	\begin{itemize}
		\item	
	\end{itemize}
\end{CHPT_SUMM_AUTO}

\begin{CHPT_SUMM_AUTO}[label = {L.-9c}]{9c. Components of the Interest Rate: No Inflation or Default Risk}
	\begin{itemize}
		\item	
	\end{itemize}
\end{CHPT_SUMM_AUTO}

\begin{CHPT_SUMM_AUTO}[label = {L.-9d}]{9d. Components of the Interest Rate: no Inflation but with Default Risk}
	\begin{itemize}
		\item	
	\end{itemize}
\end{CHPT_SUMM_AUTO}

\begin{CHPT_SUMM_AUTO}[label = {L.-9e}]{9e. Components of the Interest Rate: Known Inflation}
	\begin{itemize}
		\item	
	\end{itemize}
\end{CHPT_SUMM_AUTO}

\begin{CHPT_SUMM_AUTO}[label = {L.-9f}]{9f. Components of the Interest Rate: Uncertain Inflationo}
	\begin{itemize}
		\item	
	\end{itemize}
\end{CHPT_SUMM_AUTO}

\begin{CHPT_SUMM_AUTO}[label = {L.-9g}]{9g. Savings and Lending Interest Rates}
	\begin{itemize}
		\item	
	\end{itemize}
\end{CHPT_SUMM_AUTO}

\begin{CHPT_SUMM_AUTO}[label = {L.-9h}]{9h. Government and Corporate Bonds}
	\begin{itemize}
		\item	
	\end{itemize}
\end{CHPT_SUMM_AUTO}

\begin{CHPT_SUMM_AUTO}[label = {L.-9i}]{9i. The Role of Central Banks}
	\begin{itemize}
		\item	
	\end{itemize}
\end{CHPT_SUMM_AUTO}